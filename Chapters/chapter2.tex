\subsection{Diffraction in Crystals} \label{chap2}

\subsubsection*{Geometric Structure Factor} 

The structure factor gives the amplitude of a scattered wave arising
from the atoms with a single primitive cell. 
\begin{equation}
\mathrm{\Phi_k} = \sum_{j} f_j(\mathrm{K)}e^{i\mathrm{K \cdot d}}
\end{equation}
For crystals composed of only one type of atom, it’s common to split
the structure factor into two parts:
\begin{equation}
\mathrm{\Phi_k} = f_j\mathrm(K)S_{\mathrm{K}}
\end{equation}
,where $f_j$ is atomic form factor and $S_\mathrm{K}$ is geometric structure factor. Now let's focus on the second one.
\begin{equation}
S_\mathrm{K} = \sum_{j = 1}^{n} e^{i \mathrm{Kd}}
\end{equation}

For a perfect crystal the lattice gives the reciprocal lattice, which determines the positions (angles) of diffracted beams, and the basis gives the structure factor $F_{hkl}$ which determines the amplitude and phase of the diffracted beams:
\begin{equation}
F_{hkl} = \sum_{j = 1}^{N} f_j \mathrm{e}^{[-2\pi i(hx_j + ky_j + lz_j)]}
\end{equation}

As seen in \autoref{chap1} a FCC unit cell contains 4 atoms, one at the origin $ x_j, y_j, z_j = (0, 0, 0)$ and one at the three adjacent face centers $ x_j, y_j, z_j = \Big(\frac{1}{2}, \frac{1}{2} , 0\Big), \Big(0, \frac{1}{2} , \frac{1}{2}\Big)\ \mathrm{and}\  \Big(\frac{1}{2} , 0 , \frac{1}{2}\Big) $

$$
F_{hkl} = \sum_{j = 1}^{4} f_j \mathrm{e}^{[-2\pi i(hx_j + ky_j + lz_j)]} = f[1 + (-1)^{h+k} + (-1)^{k+l} + (-1)^{h+l}]
$$

with the result

$$
F_{hkl} = 
\left\{ \begin{array}{ll}
4f, & h,k,l\ $all even or all odd$ \\
0, & h,k,l\ $mixed parity$\\
\end{array} \right.
$$

The planes of the indices for which the geometric structure factor is zero
lead to systemetic extinctions.

\subsubsection*{Diffraction Maximums}
Bragg diffraction occurs when radiation, with a wavelength comparable to atomic 
spacings, is scattered in a specular fashion by the atoms of a crystalline system, and 
undergoes constructive interference. For a crystalline solid, the waves are scattered 
from lattice planes separated by the interplanar distance $d$. Bragg's law,  describes 
the condition on for the constructive interference to be at its strongest by formula:
\begin{equation}
\label{Bragg}
2d_{hkl} \sin{\theta} =  \lambda
\end{equation}
We need to use the concept of the reciprocal lattice to evaluate the lattice structure
factor $S$, which is involved in the x-ray scattering process. In equation:
\begin{equation}
S = G_{hkl}
\end{equation}
The scattering vector s is equal to a
reciprocal lattice vector.Also this implies that $s$ is normal to the $(hkl)$ crystal planes. For a cubic system we gave also a formula:
\begin{equation}
\frac{1}{d^2} = \frac{h^2 + k^2 + l^2}{a^2}
\end{equation}
After few transformation we get:
$$
d_{hkl} = \frac{a}{\sqrt{h^2 + l^2 + k^2}}
$$

Planes where we obtain a first 3 difraction maximums are following: 
(These are the first three planes which meet the condition of all
indices even or all odd)

$$P_1 = (111) \qquad P_2 = (200) \qquad P_3 = ( 220) $$ 

We can calculate interplanar distances:

$$d_{111} = \frac{5,58}{\sqrt{1^2 + 1^2 + 1^2}} = 3,23 \AA$$
$$d_{200} = \frac{5,58}{\sqrt{2^2 + 0^2 + 0^2}} = 2,79 \AA$$
$$d_{220} = \frac{5,58}{\sqrt{2^2 + 2^2 + 0^2}} = 1,98 \AA$$

Now we can calculate the Bragg angles for the corresponding maximas.
For the X-Ray wavelength $\lambda = 1.54 \, \mathring{A}$

$$
sin \theta = \frac{\lambda}{2d}
$$
$$
\theta = \arcsin{\frac{2d}{\lambda}}
$$

$$
\theta_1 = 13,3^\circ \qquad \theta_2 = 15,7^\circ \qquad \theta_3 = 22,3^\circ
$$

These are the Bragg angles for wich maximums appear in an X-ray diffraction pattern.


\subsubsection*{Reciprocal Lattice}
In order to identify the reciprocal lattice we start by the calculating
the reciprocal lattice vectors, which form a new set of basis vectors.

As in \autoref{chap1} already explained the three primitive vectors
of the FCC structure are:

$$\vec{u} = \frac{a}{2} \left(\begin{matrix}1\\1\\0\\\end{matrix}\right) \qquad
\vec{v} = \frac{a}{2} \left(\begin{matrix}0\\1\\1\\\end{matrix}\right) \qquad
\vec{w} = \frac{a}{2} \left(\begin{matrix}1\\0\\1\\\end{matrix}\right)$$

The new reciprocal lattice vectors are defined as:

\begin{equation}
    \vec{u^*} = \frac{2 \pi}{V_{PC}} (\vec{v} \times \vec{w}) \qquad
    \vec{v^*} = \frac{2 \pi}{V_{PC}} (\vec{w} \times \vec{u}) \qquad
    \vec{w^*} = \frac{2 \pi}{V_{PC}} (\vec{u} \times \vec{v})
\end{equation}

With $V_{PC}$ the volume of the primitive cell given from \autoref{chap1}
as:
$$V_{PC} = \frac{a^3}{4}$$

The three vectors result in:

$$
    \vec{u^*} = \frac{2\pi}{a} \left(\begin{matrix}1\\1\\-1\\\end{matrix}\right) \qquad
    \vec{v^*} = \frac{2\pi}{a} \left(\begin{matrix}-1\\1\\1\\\end{matrix}\right) \qquad
    \vec{w^*} = \frac{2\pi}{a} \left(\begin{matrix}1\\-1\\1\\\end{matrix}\right)
$$

Which represents the primitive translation vectors of a BCC lattice.
Therefore the bcc lattice is the reciprocal lattice to the FCC lattice
in which calcium crystallizes.


\subsubsection*{Edwald Sphere}

If we draw the wavevector $\mathbf{k}$ in the reciprocal lattice and let it
terminate at any reciprocal lattice point.
We draw a sphere of the radius $k=2\pi/ \lambda$ about the origin of $\mathbf{k}$
A diffracted beam will be formed if this sphere intersects any other point in the 
reciprocal lattice.
The $\mathbf{G_{\overline{2}\overline{1}0}}$ vector represent the 
$(\overline{2}\overline{1}0)$ plane.

\begin{figure}[H]
	\centering
	\includegraphics[width=0.5\linewidth]{Graphics/Chapter2/ewald_sphere.png}
	\caption{Ewald Sphere}
	\label{fig:ewals_sphere}
\end{figure}

Due Geometric to considerations the following relation can be derived.

$$\frac{G_{\overline{2}\overline{1}0}}{2a^*} = \frac{2 k_0}{G_{\overline{2}\overline{1}0}} \qquad \Rightarrow k_0 = \frac{5a^*}{4} = \frac{5\pi}{2a}$$

As for the wavelength $\lambda$  (with $a = 5.56 \, \mathring{A}$):

$$\lambda = \frac{2\pi}{k_0} = \frac{4a}{5} = 4.45  \, \mathring{A}$$

As for the angle $2\theta$ follows

$$2\theta = 180^\circ - 2 \cos^{-1} \left( \frac{G_{\overline{2}\overline{1}0}}{2k_0} \right) =
    180^\circ - 2 \cos^{-1} \left( \frac{\sqrt{5}}{10} \right) = 25.84 ^\circ$$