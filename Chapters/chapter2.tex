\section{Chapter 2}

\textbf{Geometric structure factor}

The structure factor gives the amplitude of a scattered wave arising
from the atoms with a single primitive cell. 
\begin{equation}
\mathrm{\Phi_k} = \sum_{j} f_j(\mathrm{K)}e^{i\mathrm{K \cdot d}}
\end{equation}
For crystals composed of only one type of atom, it’s common to split
the structure factor into two parts:
\begin{equation}
\mathrm{\Phi_k} = f_j\mathrm(K)S_{\mathrm{K}}
\end{equation}
,where $f_j$ is atomic form factor and $S_\mathrm{K}$ is geometric structure factor. Now let's focus on the second one.
\begin{equation}
S_\mathrm{K} = \sum_{j = 1}^{n} e^{i \mathrm{Kd}}
\end{equation}
\begin{displaymath}
Calculations/needed
\end{displaymath}
\begin{displaymath}
systematic/extinctions/needed
\end{displaymath}

\textbf{Diffraction}

Bragg diffraction occurs when radiation, with a wavelength comparable to atomic spacings, is scattered in a specular fashion by the atoms of a crystalline system, and undergoes constructive interference. For a crystalline solid, the waves are scattered from lattice planes separated by the interplanar distance $d$. Bragg's law,  describes the condition on for the constructive interference to be at its strongest by formula:
\begin{equation}
2d \sin{\theta} = n \lambda
CALCULATIONS
\end{equation}

\textbf{Reciprocal lattice}

Starting with a lattice whose basis vactors are $\textbf{a}$, $\textbf{b}$, and $\textbf{c}$, we can define a new set of basis vectors $\textbf{a'}$, $\textbf{b'}$ and $\textbf{c'}$. according to the relations:
\begin{equation}
\textbf{a'} = \frac{2 \pi}{\Omega_c} (\textbf{b} \times \textbf{c})
\end{equation}
\begin{equation}
\textbf{b'} = \frac{2 \pi}{\Omega_c} (\textbf{c} \times \textbf{a})
\end{equation}
\begin{equation}
\textbf{c'} = \frac{2 \pi}{\Omega_c} (\textbf{a} \times \textbf{b})
\end{equation}
,where $\Omega_c = \textbf{a} \cdot (\textbf{b} \times \textbf{c})$. So we can now calculate vectors of reciprocal lattice:

$$
\textbf{a'}= \frac{2\pi}{a} \left(\begin{matrix}1\\1\\0\\ \end{matrix}\right)	\qquad
\textbf{b'}= \frac{2\pi}{a} \left(\begin{matrix}0\\1\\1\\ \end{matrix}\right)	\qquad
\textbf{c'}= \frac{2\pi}{a} \left(\begin{matrix}1\\0\\1\\ \end{matrix}\right)	
$$