Calcium metal is used as a reducing agent in preparing other metals such as 
thorium and uranium. 
It is also used as an alloying agent for aluminium, beryllium, copper, lead 
and magnesium alloys.
These use cases make a deeper understanding of solids states essential.

In \ref{chap1} \nameref{chap1} we will talk about the crystal structures and how they 
are described.
Most of the basic concepts will be shown directly by the example of calcium. 
The book \textit{Elementary Solid State Physics}
provided the necessary background information. In \ref{chap2} \nameref{chap2} we will 
talk about diffraction in this crystal,
as by studying the diffraction pattern of a beam information about the 
structure of the crystal can be obtained.
\textit{Elementary Solid State Physics} \cite{elementary_SSP} delivered insights in these concepts to get 
further information or different 
explanation to some topics, \textit{Introduction to Solid State Physics} \cite{kittel} was used in addition. 
In the first two parts, it is assumed that the Atoms stay at rest at their
position. In reality this is not true as atoms oscillate arround their
rest position. In \ref{chap3} \nameref{chap3} we will show some of these effects.
