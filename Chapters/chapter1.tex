\section{Chapter 1}

Calcium, as can be seen in the following table, has a
FCC structure.

\begin{figure}[H]
	\centering
	\includegraphics[width=0.7\linewidth]{Graphics/Chapter1/StructuresAndCellDimension_Table1_2_Omar}
	\caption{}
	\label{}
\end{figure}


\textbf{Unit Cell}

The convential Unit Cell of a FCC lattice looks as follows:

\begin{figure}[H]
	\centering
	\includegraphics[width=0.4\linewidth]{Graphics/Chapter1/face-centered_cubic_lattice.png}
	\caption{FCC-Lattice}
	\label{}
\end{figure}

\textbf{Primitive Vectors}

The three primitive vectors are

\begin{figure}[H]
	\centering
	\includegraphics[width=0.5\linewidth]{Graphics/Chapter1/prim_vec}
	\caption{Primitive Vectors in a FCC-Lattice}
	\label{}
\end{figure}

$$\vec{u} = \frac{a}{2} \left(\begin{matrix}1\\1\\0\\\end{matrix}\right) \qquad
  \vec{v} = \frac{a}{2} \left(\begin{matrix}0\\1\\1\\\end{matrix}\right) \qquad
  \vec{w} = \frac{a}{2} \left(\begin{matrix}1\\0\\1\\\end{matrix}\right)$$

Whith these 3 base vectors a parallelepiped is given which is a primitive cell.
The volume of the primitive cell can be calculated with the following formula

$$V_{PC} = \vert (\vec{u} \times \vec{v})  \cdot \vec{w} \vert$$

which equals (with $a = 5.56 \, \mathring{A}$)

$$V_{PC} = \frac{a^3}{4} = 4.297 \cdot 10^{-30} \,m^3 = 4.297 \cdot 10^{-24} \,cm^3$$
\textbf{Packaging Factor}

The Packaging Factor can be calculated as the ratio between the
volume of the atoms in the unit cell to the volume of the unit cell.

The volume of the unit cell can be calculated as:

$$V_{UC} = a^3$$


The unit cell containts 4 whole atoms.
One eighth of a atomic sphere at each corner (8) and one half at 
each cube face (6).

\begin{figure}[H]
	\centering
	\includegraphics[width=0.3\linewidth]{Graphics/Chapter1/r_a_relation}
	\caption{Relation between the atomic radius and the parameter a in a FFC}
	\label{fig:r_a_relation}
\end{figure}

 As \autoref{fig:r_a_relation} shows the relationship between the parameter $a$ and the radius of the atomic sphere is given as:
$$r = \frac{\sqrt{2}}{4} a $$

And further the volume of the sphere

$$V_{Atom} = \frac{4}{3} \pi r^3 = \frac{a^3 \pi}{\sqrt{2^5}\cdot 3}$$

So the Atomic Packaging Factor $APF$ can be calculated as ratio between the
volume consumed by the atoms to the whole volume.

$$APF = \frac{4 \cdot V_{Atom} }{V_{UC}} = \frac{\pi}{3 \cdot \sqrt{2}} \approx 74\%$$


\textbf{Density}

The atomic mass of calcium us given as:
$$M_{Ca} = 40.078 \frac{g}{mol}$$


$$\rho = \frac{4}{N_A} \cdot \frac{M_{Ca}}{V_{UC}} = 1.55 \frac{g}{cm^3}$$


\textbf{Planes}

In the following the planes $P1: \, (0\overline{3}2)$ and $P2: \,(\overline{1}21)$ are drawn inside the unit cell.
The Miller-Indices of the planes corresbond to the following plane equations:
$$P1: \quad -\frac{1}{3} y + \frac{1}{2} z = 1$$
$$P2: \quad -x +\frac{1}{2} y + z = 1$$

With respect of the fact that all parallel planes have the same Miller-Indices the planes which were drawn are:
$$P1: \quad z = \frac{2}{3}y$$
$$P2: \quad z = x -\frac{1}{2}y$$

\begin{figure}[H]
	\centering
	\includegraphics[width=0.6\linewidth]{Graphics/Chapter1/PLANE032}
	\caption{$(0\overline{3}2)$-Plane in a FCC-Lattice}
	\label{}
\end{figure}


\begin{figure}[H]
	\centering
	\includegraphics[width=0.6\linewidth]{Graphics/Chapter1/PLANE121}
	\caption{$(\overline{1}21)$-Plane in a FCC-Lattice}
	\label{}
\end{figure}


\textbf{Linear Density [110]}

\begin{figure}[H]
	\centering
	\includegraphics[width=0.5\linewidth]{Graphics/Chapter1/Lin_Den}
	\caption{Linear Density of FCC in [110] Direction}
	\label{Lin_Den}
\end{figure}

\autoref{Lin_Den} shows the [110] direction in a FCC lattice.
As you can see the [110] direction includes 2 atoms inside a 
length of $\sqrt{2}a$.

Therefore  (with $a = 5.56 \, \mathring{A}$)

$$\lambda = \frac{2 \, Atoms}{\sqrt{2}a} = \frac{\sqrt{2}}{5.56} \frac{Atoms}{\mathring{A}}$$

\textbf{Potential Energy}

The potential energy between to adjacent ions can be represented by

\begin{equation}
	E(r) = - \frac{A}{r} + \frac{B}{r^n}
	\label{eq:Pot_Energy}
\end{equation}


To calculate the bonding energy $E_0 = E(r_0)$, which is a minimum of the function $E(r)$,
the derivative has to equals zero.
The negative derivative of the bonding energy equals the interatomic force.

$$F(r) = - \frac{\partial E(r)}{\partial r} = 0$$

$$-\frac{A}{r^2} + \frac{nB}{r^{n+1}} = 0$$
$$\Rightarrow r_0 = \left( \frac{A}{nB} \right)^{\frac{1}{n-1}}$$

By inserting the result for $r_0$ into \autoref{eq:Pot_Energy}, the bonding energy $E_0$ in terms of $A$, $B$ and $n$ results as:

$$E_0 = E(r_0) = - \frac{A}{\left( \frac{A}{nB} \right)^{\frac{1}{n-1}}} + 
				\frac{B}{\left( \frac{A}{nB} \right)^{\frac{n}{n-1}}}$$

