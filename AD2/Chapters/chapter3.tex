\subsection{Chap 3} \label{chap3}

\subsubsection*{Question 1}
The carrier distribution function in CB für an intrinsic semiconductor
is given as the product:

$$g_e(E) f(E)$$

with

$$g_e(E) = \frac{1}{2\pi^2} \left( \frac{2m_e^*}{\hbar} \right)^\frac{3}{2} (E-E_C)^\frac{1}{2}$$

$$f(E) = \frac{1}{e^{(E-E_F)/k_BT}+1}  \approx e^{-\frac{E-E_F}{k_BT}}$$

So for the maximum 

$$\frac{d}{dE} \left(\frac{1}{2\pi^2} \left( \frac{2m_e^*}{\hbar} \right)^\frac{3}{2} (E-E_C)^\frac{1}{2} e^{-\frac{E-E_F}{k_BT}} \right) = 0$$

By dividing through the constant factors:

$$\frac{d}{dE} \left( (E-E_C)^\frac{1}{2} e^{-\frac{E}{k_BT}} \right) = 0$$

$$\frac{1}{2} (E-E_C)^{-\frac{1}{2}} e^{-\frac{E}{k_BT}} + \frac{-1}{k_BT} (E-E_C)^\frac{1}{2} e^{-\frac{E}{k_BT}} = 0$$

And finaly

$$E = E_C + \frac{k_BT}{2}$$


\subsubsection*{Question 2}

For the effective density of states in theh conducition beand 
the following relation is known:

$$ N_c = 2 \left( \frac{m_e^*k_BT}{2\pi\hbar^2}\right)^{\frac{3}{2}}$$

Which lead to the following result
$$ N_c = 2.415 \cdot 10^{24} \frac{1}{m^3}$$

In the same manner also the density of states in the valence band can be
calculatetd.

$$ N_v = 2 \left( \frac{m_h^*k_BT}{2\pi\hbar^2}\right)^{\frac{3}{2}}$$
$$ N_v = 1.796 \cdot 10^{25} \frac{1}{m^3}$$

In an intrinsic semiconductor the concentration of the holes and the 
electron are equal and this concentration is named as
intrinsic carrier's concentration. 

Which is knwon to be:
$$n_i = p_i = 2 \left( \frac{k_BT}{2\pi\hbar^2} \right)^{\frac{3}{2}}
  (m_e^*m_h^*)^{\frac{3}{4}} e^{-\frac{E_g}{2k_BT}}$$
$$n_i = p_i = 0.959 \cdot 10^3 \frac{1}{m^3}$$

For the intrinsic Fermi can be calculatetd as 

$$E_{Fi} = \frac{E_g}{2} + \frac{3}{4} k_B T \ln\left( \frac{m_h^*}{m_e^*} \right)$$

$$E_{Fi} = 1.326 eV$$


\subsubsection*{Question 3}

If the given semiconductor CdS was doped p-type with a
concentration of $10^{15}$ acceptor impurities the following
about the concentration and holes at $T=0\degree K$ can be said.

At this point the thermal energy becomes too small to cause electron exication, which means that all electrons fall from
the conducition Band into the donor level. Also the conductivity
goes to zero. This process is called freeze out. 
So for the concentration of electron and holes at $T=0 \degree K$

The concentration of the elctrons:

$n(T=0 \degree K) = 0$

As the semiconductor is doped with holes the concentration of the
holes is the same as it was initaly.

$p(T=0 \degree K) = 10^{15}cm^{-3}$


\subsubsection*{Question 4}

As the concentration of acceptor impurities is much higher then
the holes concentration of the intrinsic semiconductor 
$(10^{15} \frac{1}{cm^3} \gg p_i)$

The concentration of the holes in the semiconductor is equal to concentration
of the impurities.

$$p = 10^{15} \frac{1}{cm^3} = 10^{18} \frac{1}{m^3}$$

As the square of the intrinsic concentrationn$n_i$ is equal to the product
of the sum of the concentration of the holes and the concentration of the
electrons.

$$np = n_i^2$$

$$n = \frac{n_i^2}{p} = \frac{(0.959 \cdot 10^3 \frac{1}{m^3})^2}{10^{18} \frac{1}{m^3}} = 9.197 \cdot 10^{-13} \frac{1}{m^3}$$

$$n = n_i e^{\frac{E_F-E_{Fi}}{k_BT}}  \qquad p = n_i e^{-\frac{E_F-E_{Fi}}{k_BT}}$$

$$(E_F-E_{Fi}) = \ln\left( \frac{n}{n_i} \right) \cdot k_BT$$

$$(E_F-E_{Fi}) = -0.896 eV $$

\subsubsection*{Question 5}

$$\epsilon_r = 8.9$$


$$
E_a = \frac{1}{\epsilon_r^2} \left(\frac{m_h}{m_0}\right) \underbrace{\left[\frac{e^4m_0}{2(4\pi\epsilon_h\hbar)^2}\right]}_{13.6eV}
$$

$$E_a =  0.14eV$$

\subsubsection*{Question 6}

\subsubsection*{Question 7}