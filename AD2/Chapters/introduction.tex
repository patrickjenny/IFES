Cadmium sulfide (CdS) is in the class II-VI semiconductor.

In \ref{chap1} \nameref{chap1} we will use the free elecon model 
to obtain some results for a given metal and discribe some of the
basic concepts. In the model it is assumed that the metal contains 
a large concentration of essentialy free electrons. These assumption
leads to an satisfying description of many effects in metal, which are of great importance in our daily live.

In \ref{chap2} \nameref{chap2} 
\textit{Elementary Solid State Physics} \cite{elementary_SSP} delivered insights in these concepts to get 

 In \ref{chap3} \nameref{chap3} 

 Semiconductor devices can display a range of useful properties, such as passing current more easily in one direction than the other, showing variable resistance, and sensitivity to light or heat. Because the electrical properties of a semiconductor material can be modified by doping, or by the application of electrical fields or light, devices made from semiconductors can be used for amplification, switching, and energy conversion.
