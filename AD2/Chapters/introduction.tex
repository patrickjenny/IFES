In \ref{chap1} \nameref{chap1} we will use the free electron model to obtain some results for a given metal and describe some of the basic concepts. In the model it is assumed that the metal contains a large concentration of essentially free electrons. These assumption leads to an satisfying description of many effects in metal, which are of great importance in our daily live.

Although the free electron model can be used to describe many phenomena in metals, the crystal
potential is neglected, which leads to an oversimplified model. To explain the experimental results quantitatively it cannot be entirely disregarded. Also, some effects cannot be explained at all without taking this potential into account. In \ref{chap2} \nameref{chap2} these models are discussed and some basic calculation are shown. \textit{Elementary Solid State Physics} \cite{elementary_SSP} delivered insights in these concepts.

Semiconductor devices can display a range of useful properties, such as passing current more easily in one direction than the other, showing variable resistance, and sensitivity to light or heat. Because the electrical properties of a semiconductor material can be modified by doping, or by the application of electrical fields or light, devices made from semiconductors can be used for amplification, switching, and energy conversion. In \ref{chap3} \nameref{chap3} we will show some of the basic concepts of semiconductors. The calculations are shown at the example of Cadmium sulfide (CdS), which is a class II-VI semiconductor.  


For Cadmium sulfide the following parameters were known:
(which were used in \ref{chap3} \nameref{chap3} )

\begin{table}[h]
    \centering
    \begin{tabular}{l|l}
     $a \, [\mathring{A}]$      & 5.82 \\ \hline
     $\epsilon_r$               & 8.9  \\ \hline
     $\mu_e \, [cm^2/Vs]$       & 340  \\ \hline
     $\mu_h \, [cm^2/Vs]$       & 340  \\ \hline
     $E_g \, [eV]$              & 2.6  \\ \hline
     $m_e/m_0$                  & 0.21 \\ \hline
     $m_h/m_0$                  & 0.80 \\ 
    \end{tabular}
    \caption{Parameters for CdS}
\end{table}
