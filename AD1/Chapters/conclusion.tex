This report described Solid-State properties of Calcium. Basic issues describing calcium have been determined.

The crystal structure of Calcium was identified as a face centered cubic structure.
The primitive vectors are described in \autoref{eq:prim_vec}

By caclulating the density of the solid the same result was achieved
as cit is found in the literature \cite{web_elem_calcium}

$$\rho_{Ca} =  1.55 \frac{g}{cm^3}$$

In \ref{chap2} \nameref{chap2} the planes that give rise to the firs 3
diffraction maximums, were identified and the Bragg angle for which they
appeaer in a X-Ray ($\lambda = 1.54 \, \mathring{A}$) diffraction pattern.
$$P_1 = (111) \qquad \theta_1 = 13,3^\circ$$
$$P_2 = (200) \qquad \theta_2 = 15,7^\circ$$
$$P_3 = (220) \qquad \theta_3 = 22,3^\circ$$
 
It could also be shown that the reciprocal lattice of FCC is a BCC lattice.
To determine which X-ray wavelength should be used to determine a maximum diffraction (spot) there was a need to use Ewald sphere issue. 

In \ref{chap3} \nameref{chap3} a expression for the density of states
in a solid (\autoref{eq:g_ome_3d}) was derived.
With this result also a expression for the Debye frequency could be calculated
(\autoref{eq:debye_frequency_int})
In \autoref{fig:disp_relat_calcium} the theoretical curve for the dispersion 
relation for Calcium is presented.

Whole report helps to undarstand basic assumptions of solid state physics.